\documentclass[14pt]{extarticle}

\usepackage{fontspec}
\usepackage{geometry}
\geometry{a4paper, left=30mm, right=10mm, top=20mm, bottom=20mm}
\usepackage[hidelinks]{hyperref}
\usepackage{polyglossia}
\usepackage{titlesec}
\usepackage{pgfplots}
\usepackage{amsmath}
\usepackage{amssymb}
\usepackage{array}
\usepackage{tikz}
\usepackage{pgf}
\usepackage{xcolor}
\usepackage{breqn}
\usepackage{natbib}
\usepackage{enumitem}

\usepackage{setspace}
\onehalfspacing

\setlength{\parindent}{12.7mm}
\setlength{\parskip}{8pt}

\usepackage{ragged2e}
\raggedbottom
\sloppy
\justifying

\bibliographystyle{plain}
\setmainfont{Times New Roman}[size=14pt]
\setdefaultlanguage{russian}

\begin{document}
\thispagestyle{empty}

\noindent \begin{minipage}{0.15\textwidth}
	\includegraphics[width=\linewidth]{img/b_logo}
\end{minipage}
\begin{minipage}{0.8\textwidth}\centering
\small
\textbf{Министерство науки и высшего образования Российской Федерации}\\
\textbf{Федеральное государственное бюджетное образовательное учреждение высшего образования}\\
\textbf{«Московский государственный технический университет имени Н.Э.~Баумана}\\
\textbf{(национальный исследовательский университет)»}\\
\textbf{(МГТУ им. Н.Э.~Баумана)}
\end{minipage}

\noindent\rule{18cm}{0pt}
\newline\newline
\noindent ФАКУЛЬТЕТ $\underline{\text{«Робототехника и комплексная автоматизация»}}$ \newline\newline
\noindent КАФЕДРА $\underline{\text{«Системы автоматизированного проектирования»}}$\newline\newline\newline
\begin{center}
	\noindent\begin{minipage}{\textwidth}\centering
        \Large\textbf{РАССЧЕТНО-ПОЯСНИТЕЛЬНАЯ ЗАПИСКА}\newline
        \large\textbf{К КУРСОВОЙ РАБОТЕ ПО ТЕМЕ:}\newline
		\textbf{По дисциплине: «Базы данных»}\newline\newline
	\end{minipage}
\end{center}

\noindent\textbf{Студент} \hfill $\underline{\text{Губанков Н.Г.}}$\newline\newline
\noindent\textbf{Группа} \hfill $\underline{\text{РК-43Б}}$\newline\newline
\noindent\textbf{Тема} \hfill $\underline{\text{Разработка информационной системы}}$\newline\newline

\noindent\text{Студент}
\hfill \underline{\makebox[4cm][r]{}}
\hspace*{1cm} \underline{\makebox[4cm][r]{Губанков Н.Г.}} \\
\small
\vspace{-20pt}
\begin{flushright}
    \textit{подпись, дата} \hspace*{4cm}
    \textit{фамилия, и. о.}
\end{flushright}

\noindent\text{Преподаватель}
\hfill \underline{\makebox[4cm][r]{}}
\hspace*{1cm} \underline{\makebox[4cm][r]{Пивоварова Н.В.}} \\
\small
\vspace{-20pt}
\begin{flushright}
    \textit{подпись, дата} \hspace*{4cm}
    \textit{фамилия, и. о.}
\end{flushright}

\normalsize
\begin{center}
	\vfill
	Москва,~\the\year
\end{center}
\clearpage


\tableofcontents
\clearpage

\section*{Аннотация}
В работе представлены артефакты проектирования информационной
системы для магазина по управлению кинотеатром.
На основании изучения предметной области выделены варианты
использования системы, определены конечные пользователи и разработана
UML диаграмма вариантов использования. Каждый вариант использования
детализирован до пользовательского сценария и системной диаграммы
последовательности и программной архитектуры.

\section*{Техническое задание}
Техническое задание на этап проектирования
Техническое задание на этап проектирования состоит из следующих пунктов:
\begin{enumerate}
    \item Определить конечных пользователей будущей системы.
    \item Составить UML-диаграмму вариантов использования.
    \item Выделить основной вариант использования информационной системы (основной
        бизнес-процесс в предметной области).
    \item Разработать систему авторизации пользователей ИС.
    \item Разработать системную архитектуру ИС.
    \item Для всех вариантов использования разработать главные успешные сценарии и
        расширения к ним.
    \item Разработать системные UML-диаграммы последовательности для всех сценариев
        использованием MVC-паттерна.
    \item Разработать требования ко всем шаблонам для каждого варианта использования.
    \item Разработать инфологическую модель предметной области в форме UML-диаграммы классов.
    \item Разработать логическую модель будущей базы данных.
\end{enumerate}

\section*{Техническое задание на этап реализации}
Техническое задание на этап реализации состоит из следующих задач:
\begin{enumerate}
    \item 
    \item Реализовать разработанную на этапе проектирования информационную систему
        на языке Python в среде фреймворка Flask.
    \item Каждый вариант использования оформить, как blueprint.
    \item Доступ конечных и внешних пользователей к вариантам использования
        реализовать с помощью декораторов.
\end{enumerate}

\clearpage
\section*{Конечные пользователи информационной системы}
В данной системе пользователи разделяются на следующие категории:
\begin{enumerate}
    \item Кассир \textit{(внешний пользователь)}
        \begin{itemize}
            \item Авторизация
            \item Продажа билетов
        \end{itemize}
    \item Менеджер \textit{(внутренний пользователь)}
        \begin{itemize}
            \item Авторизация
            \item Продажа билетов
            \item Работа с запросами
        \end{itemize}
    \item Администратор \textit{(внутренний пользователь)}
        \begin{itemize}
            \item Авторизация
            \item Продажа билетов
            \item Работа с запросами
            \item Работа с отчетами
        \end{itemize}
\end{enumerate}

\begin{figure}[h!]
    \includegraphics[width=\linewidth]{build/img/scenarios.png}
    \caption{UML-диаграмма вариантов использования}
    \label{fig:scenarios}
\end{figure}

\clearpage

\section*{Главное меню}
Пункты главного меню содержат все варианты использования и пункт для выхода
из системы. При запуске системы управление передается контроллеру главного
меню.
Пункты главного меню для внутренних пользователей:
\begin{itemize}
    \item Покупка билетов;
    \item Работа с запросами;
    \item Работа с отчетами;
    \item Выход из аккаунта (системы).
    \item Пункты главного для внешних пользователей:
    \item Выход из аккаунта (системы).
\end{itemize}
\paragraph{Предусловия:} пользователь авторизировался.
\paragraph{Гарантия:} при наличии прав на использование выбранного пункта меню
пользователь переходит на страницу соответствующего варианта использования.
\paragraph{Минимальная гарантия:} пользователь остаётся в системе с сохранением состояния
сессии.
\paragraph{Сценарий работы главного меню:}
\begin{enumerate}
    \item Пользователь запускает сценарий;
    \item Система выдает главное меню;
    \item Пользователь выбирает один из пунктов (вариантов использования);
    \item Система перелает управление контроллеру соответствующего варианта использования.
\end{enumerate}
\clearpage
\paragraph{Исключения:}
\begin{enumerate}[label=\arabic*A.]
    \item Пользователь не авторизован.
\end{enumerate}

\begin{figure}[h!]
    \includegraphics[width=\linewidth]{build/img/main_menu.png}
    \caption{UML-диаграмма главного меню}
    \label{fig:main_menu}
\end{figure}

\begin{figure}[h!]
    \includegraphics[width=\linewidth]{img/main_menu.png}
    \caption{Програмная архитектура варианта использования}
    \label{fig:fs_main_menu}
\end{figure}

\paragraph{Требования к шаблонам HTML:} Динамический шаблон «Главное меню».
Шаблон показывает главное меню со всеми пунктами – вариантами использования и
информацию об авторизации.
Шаблон содержит ссылки:
\begin{itemize}
    \item На контроллер работы с запросами (адрес «blueprints/queries_menu»)
    \item На контроллер работы с отчётами (адрес «blueprints/ticket_report»)
    \item На контроллер бизнес-процесса (адрес «blueprints/ticket_cart»)
    \item На выход из системы (адрес «/logout»)
\end{itemize}

\clearpage
\section*{Авторизация}
Авторизация создана для групп пользователей: внешних (кассиры) и
внутренних (администраторы и менеджеры).

У этих групп разные права доступа к остальным вариантам использования.
Внутренние пользователи имеют доступ к работе с запросами и работе с отчетами,
внешние пользователи имеют доступ к запросам и бизнес-процессу.
\paragraph{Предусловия:} пользователю передана форма для ввода логина и пароля.
\paragraph{Гарантия:} пользователь вошел в систему и получил соответствующую роль.
\paragraph{Минимальная гарантия:} пользователь получил сообщение, о неверно введенном
логине или пароле, и кнопку для продолжения работы.
\paragraph{Сценарий работы авторизации:}
\begin{enumerate}
    \item Пользователь запускает сценарий;
    \item Система присылает форму для ввода логина и пароля;
    \item Пользователь вводит данные;
    \item Система получает данные из формы;
    \item Система успешно производит аутентификацию и выдает пользователю главноe меню
\end{enumerate}
\paragraph{Исключения:}
\begin{enumerate}[start=5,label=\arabic*A.]
    \item Пользователь ввел неверные данные
\end{enumerate}

\begin{figure}[h!]
    \includegraphics[width=\linewidth]{build/img/auth.png}
    \caption{UML-диаграмма авторизации}
    \label{fig:auth}
\end{figure}

\begin{figure}[h!]
    \includegraphics[width=\linewidth]{img/auth.png}
    \caption{Програмная архитектура варианта использования}
    \label{fig:fs_auth}
\end{figure}

\paragraph{Требования к шаблонам html:} Динамический шаблон «Авторизация».
Шаблон предназначен для ввода идентификационных данных пользователя и
последующей авторизации в системе.
Шаблон содержит два поле типа input для ввода логина и пароля. Обе формы
являются обязательными для заполнения.
Под формами должна располагаться кнопка «Войти» типа submit, осуществляющая
передачу данных на сервер с последующей проверкой успешной авторизации.
Шаблон содержит динамическую вставку для сообщения об ошибке, если имеется
сообщение об ошибке.

\clearpage
\section*{Работа с запросами}
\subsection*{Меню запросов}
Работа с запросами предназначена для внешних и внутренних пользователей.
\paragraph{Предусловия:} пользователю выдана главная страница.
\paragraph{Гарантия:} пользователь перешел на страницу «Меню запросов».
\paragraph{Минимальная гарантия:} пользователь получил сообщение, если у него не хватает
прав доступа или он не авторизирован, и кнопку для продолжения работы.
\paragraph{Сценарий работы меню запросов:}
\begin{enumerate}
    \item Пользователь запускает сценарий;
    \item Система присылает меню запросов.
\end{enumerate}
Исключения:
\begin{enumerate}[start=2,label=\arabic*A.]
    \item У пользователя не хватает прав для работы с запросами
\end{enumerate}

\begin{figure}[h!]
    \includegraphics[width=\linewidth]{build/img/queries_menu.png}
    \caption{UML-диаграмма меню запросов}
    \label{fig:queries_menu}
\end{figure}

\begin{figure}[h!]
    \includegraphics[width=\linewidth]{img/queries_menu.png}
    \caption{Програмная архитектура варианта использования}
    \label{fig:fs_queris_menu}
\end{figure}

\paragraph{Требования к шаблонам HTML:} Динамический шаблон «Меню запросов».
Шаблон показывает меню запросов со всеми пунктами – различными типами запросов.
Шаблон содержит ссылки:
\begin{itemize}
    \item На контроллер запросов поиска фильмов (адрес «blueprints/films_search»)
    \item На контроллер запросов доступных мест (адрес «blueprints/available_seats»)
    \item На контроллер запросов списка залов (адрес «blueprints/halls_list»)
    \item На контроллер главного меню (адрес «/»)
\end{itemize}

\clearpage
\subsection*{Запросы}
Работа с запросами предназначена для внешних и внутренних пользователей.
\paragraph{Предусловия:} пользователь авторизированф и имеет права для работы с запросами.
\paragraph{Гарантия:} пользователь получает результат запроса.
\paragraph{Минимальная гарантия:} пользователь получил сообщение, если в БД не было
данных в результате выполнения запроса или пользователь некорректно ввел
данные, и кнопку для продолжения работы.
\paragraph{Сценарий использования:}
\begin{enumerate}
    \item Пользователь запускает сценарий;
    \item Система выдает форму для ввода для параметров;
    \item Пользователь производит ввод;
    \item Система выполняет запрос и возвращает пользователю страницу с
        результатом данного запроса в виде таблицы.
\end{enumerate}
\paragraph{Исключения:}
\begin{enumerate}[start=3,label=\arabic*A.]
    \item Пользователь ввёл некорректные данные, и система выдаёт сообщение об
        ошибочном вводе пользователю и кнопку возврата к главному меню или
        попробовать заново;
    \item Пользователь ввёл данные, которых нет в БД, и система оповещает об
        отсутствии информации по данному запросу и предоставляет кнопку возврата к
        главному меню или попробовать заново.
\end{enumerate}

\begin{figure}[h!]
    \includegraphics[width=\linewidth]{build/img/query.png}
    \caption{UML-диаграмма запроса}
    \label{fig:query}
\end{figure}

\begin{figure}[h!]
    \includegraphics[width=\linewidth]{img/query.png}
    \caption{Програмная архитектура варианта использования}
    \label{fig:fs_query}
\end{figure}

\clearpage
\section*{Работа с отчетами}
\subsection*{Меню работы с отчетами}
Работа с отчётами предназначена для внутренних пользователей.
\paragraph{Предусловия:} пользователю выдана главная страница с кнопками просмотра отчёта
и запроса отчёта.
\paragraph{Гарантия:} пользователь перешел на страницу «Меню отчётов».
\paragraph{Минимальная гарантия:} пользователь получил сообщение, если у него не хватает
прав доступа или он не авторизирован, и кнопку для продолжения работы.
\paragraph{Сценарий работы меню отчётов:}
\begin{enumerate}
    \item Пользователь запускает сценарий;
    \item Система присылает меню отчётов.
\end{enumerate}
\paragraph{Исключения:}
\begin{enumerate}[start=3,label=\arabic*A.]
    \item У пользователя не хватает прав для работы с отчётами
\end{enumerate}

\begin{figure}[h!]
    \includegraphics[width=\linewidth]{build/img/report_menu.png}
    \caption{UML-диаграмма меню работы с отчетами}
    \label{fig:report_menu}
\end{figure}

\paragraph{Требования в шаблонам html:}
\begin{itemize}
    \item Шаблон содержит форму типа select для ввода типа отчета
    \item Шаблон содержит форму типа input для ввода начальной и конечной дат.
    \item Шаблон содержит динамический блок информации с сообщением от системы.
\end{itemize}

\subsection*{Создание отчета}
\paragraph{Предусловия:} пользователю авторизирован в качестве внутреннего пользователя.
\paragraph{Гарантия:} при наличии прав доступа пользователь может создать отчёт или
запросить его.
\paragraph{Минимальная гарантия:} будет выводиться сообщение об ошибке.
\paragraph{Сценарий использования:}
\begin{enumerate}
    \item Пользователь запускает сценарий;
    \item Система выдаёт форму для ввода параметров начальной и конечной даты;
    \item Пользователь вводит данные;
    \item Система получает данные, выполняет процедуру и возвращает пользователю информационное сообщение.
\end{enumerate}
\paragraph{Исключения:}
\begin{enumerate}[start=4,label=\arabic*A.]
    \item Отчёт за данный период уже сформирован.
\end{enumerate}

\begin{figure}[h!]
    \includegraphics[width=\linewidth]{build/img/report_add.png}
    \caption{UML-диаграмма создания отчета}
    \label{fig:report_add}
\end{figure}

\subsection*{Просмотр отчета}
\paragraph{Предусловия:} пользователю авторизирован в качестве внутреннего пользователя.
\paragraph{Гарантия:} при наличии прав доступа пользователь может просмотреть отчет.
\paragraph{Минимальная гарантия:} будет выводиться сообщение об ошибке.
\paragraph{Сценарий использования:}
\begin{enumerate}
    \item Пользователь запускает сценарий;
    \item Система выдаёт форму для ввода параметров начальной и конечной даты;
    \item Пользователь вводит данные;
    \item Система получает данные, выполняет запрос и возвращает пользователю результат в виде таблицы.
\end{enumerate}
\paragraph{Исключения:}
\begin{enumerate}[start=4,label=\arabic*A.]
    \item Отчёт за данный период не сформирован.
\end{enumerate}

\begin{figure}[h!]
    \includegraphics[width=\linewidth]{build/img/report_view.png}
    \caption{UML-диаграмма просмотра отчета}
    \label{fig:report_view}
\end{figure}

\begin{figure}[h!]
    \includegraphics[width=\linewidth]{img/report.png}
    \caption{Програмная архитектура варианта использования}
    \label{fig:fs_report}
\end{figure}

\clearpage
\section*{Продажа билетов}
Бизнес-процесс создан для внешней группы пользователей.
\paragraph{Предусловия:} пользователь (кассир) успешно авторизировался в
информационной системе и обладает соответствующими правами доступа к данному
варианту использования.
\paragraph{Гарантия:} пользователь получает возможность выбрать причал, к которому
пришвартован корабль выбрать и завербовать сотрудников порта.
\paragraph{Минимальная гарантия:} пользователь получит сообщение, о том, что произошла
ошибка, и кнопку для возврата в главное меню.
\paragraph{Сценарий набора команды:}
\begin{enumerate}
    \item Пользователь запускает сценарий;
    \item Система присылает форму для выбора сеанса;
    \item Пользователь выбирает сеанс;
    \item Система присылает форму с доступными местами на сеанс;
    \item Пользователь выбирает билеты из доступных;
    \item Пользователь отправляет данные системе, выбранные билеты система заносит в Redis;
    \item Пользователь нажимает кнопку «Оформить покупку» и отправляет данные системе, система передаёт данные БД;
    \item Система выводит пользователю сообщение об успехе.
\end{enumerate}

\paragraph{Исключения:}
\begin{enumerate}[start=1,label=\arabic*A.]
    \item Пользователь не авторизован или не имеет соответсвующих прав
\end{enumerate}
\begin{enumerate}[start=7,label=\arabic*A.]
    \item Билет был куплен другим пользователем после выбора
\end{enumerate}

\begin{figure}[h!]
    \includegraphics[height=\textheight,keepaspectratio]{build/img/cart_view.png}
    \label{fig:cart}
\end{figure}

\begin{figure}[h!]
    \includegraphics[width=\linewidth]{img/ticket_cart.png}
    \caption{Програмная архитектура варианта использования}
    \label{fig:fs_cart}
\end{figure}

\paragraph{Требования к шаблону html:}
Динамический шаблон «Продажа билетов».
\begin{enumerate}
    \item Шаблон содержит следующие форму ввода select для выбора сеанса.;
    \item Шаблон содержит кнопу перехода в главное менюl
    \item Шаблон содержит поле с информацией о выбранном сеансе;
    \item Шаблон содержит множество форм, отоброжающих билеты (доступные, в корзине и проданные);
    \item Шаблон содержит таблицу с билетами в корзине и их общей ценой, а также кнопками для удаления из корзины;
    \item Шаблон содержит кнопки для оформления покупки и очистки корзины.
\end{enumerate}

\clearpage
\section*{Инфологическая модель базы данных}

\begin{figure}[h!]
    \includegraphics[width=\linewidth]{img/db_infological_smaller.png}
    \label{fig:bd_infological}
\end{figure}


\section*{Логическая модель базы данных}
\begin{table}[ht]
\centering
\begin{tabular}{|c|c|c|c|}
\hline
user_id & login & password & role \\
\hline
PK &  &  &  \\
\hline
\end{tabular}
\caption{users}
\end{table}

\begin{table}[ht]
\centering
\begin{tabular}{|c|c|c|c|}
\hline
hall_id & hall_number & name & total_seats \\
\hline
PK &  &  &  \\
\hline
\end{tabular}
\caption{halls}
\end{table}

\begin{table}[ht]
\centering
\begin{tabular}{|c|c|c|c|}
\hline
ts_id & hall_id & locale & ts_name \\
\hline
PK & FK &  &  \\
\hline
\end{tabular}
\caption{halls_ts}
\end{table}

\begin{table}[ht]
\centering
\begin{tabular}{|c|c|c|c|c|c}
\hline
schema_id & hall_id & row_num & start_seat & end_seat & base_price \\
\hline
PK & FK &  &  &  &  \\
\hline
\end{tabular}
\caption{hall_schema}
\end{table}

\begin{table}[ht]
\centering
\begin{tabular}{|c|c|c|c|c|c|c|c|}
\hline
film_id & title & country & year & director & studio & duration & description \\
\hline
PK &  &  &  &  &  &  &  \\
\hline
\end{tabular}
\caption{films}
\end{table}

\begin{table}[ht]
\centering
\begin{tabular}{|c|c|c|c|c|c|c|}
\hline
ts_id & film_id & locale & ts_country & ts_title & ts_director & ts_description \\
\hline
PK & FK &  &  &  &  &  \\
\hline
\end{tabular}
\caption{films_ts}
\end{table}

\begin{table}[ht]
\centering
\begin{tabular}{|c|c|c|c|c|c|}
\hline
session_id & film_id & hall_id & session_date & session_time & price_coef \\
\hline
PK & FK & FK &  &  &  \\
\hline
\end{tabular}
\caption{sessions}
\end{table}

\begin{table}[ht]
\centering
\begin{tabular}{|c|c|c|c|c|c|c|}
\hline
ticket_id & session_id & row_num & seat_number & price & is_sold & sold_datetime \\
\hline
PK & FK &  &  &  &  &  \\
\hline
\end{tabular}
\caption{tickets}
\end{table}

\begin{table}[ht]
\centering
\begin{tabular}{|c|c|c|c|c|}
\hline
report_id & report_month & report_year & total_tickets_sold & created_at \\
\hline
PK &  &  &  &  \\
\hline
\end{tabular}
\caption{ticket_count_reports}
\end{table}

\begin{table}[ht]
\centering
\begin{tabular}{|c|c|c|c|c|}
\hline
report_id & report_month & report_year & total_revenue & created_at \\
\hline
PK &  &  &  &  \\
\hline
\end{tabular}
\caption{total_revenue_reports}
\end{table}

\begin{table}[ht]
\centering
\begin{tabular}{|c|c|c|c|c|c|}
\hline
report_id & report_month & report_year & sessions_count & avg_ticket_price & created_at \\
\hline
PK &  &  &  &  &  \\
\hline
\end{tabular}
\caption{sessions_avgprice_reports}
\end{table}

\end{document}
