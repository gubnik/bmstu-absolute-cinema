\documentclass[14pt]{extarticle}

\usepackage{fontspec}
\usepackage{geometry}
\geometry{a4paper, left=30mm, right=10mm, top=20mm, bottom=20mm}
\usepackage[hidelinks]{hyperref}
\usepackage{polyglossia}
\usepackage{titlesec}
\usepackage{pgfplots}
\usepackage{amsmath}
\usepackage{amssymb}
\usepackage{array}
\usepackage{tikz}
\usepackage{pgf}
\usepackage{xcolor}
\usepackage{breqn}
\usepackage{natbib}
\usepackage{enumitem}

\usepackage{setspace}
\onehalfspacing

\setlength{\parindent}{12.7mm}
\setlength{\parskip}{8pt}

\usepackage{ragged2e}
\raggedbottom
\sloppy
\justifying

\bibliographystyle{plain}
\setmainfont{Times New Roman}[size=14pt]
\setdefaultlanguage{russian}

\begin{document}
\tableofcontents
\clearpage

\section*{Аннотация}
В работе представлены артефакты проектирования информационной
системы для магазина по управлению кинотеатром.
На основании изучения предметной области выделены варианты
использования системы, определены конечные пользователи и разработана
UML диаграмма вариантов использования. Каждый вариант использования
детализирован до пользовательского сценария и системной диаграммы
последовательности и программной архитектуры.

\section*{Техническое задание}
Техническое задание на этап проектирования
Техническое задание на этап проектирования состоит из следующих пунктов:
\begin{enumerate}
    \item Определить конечных пользователей будущей системы.
    \item Составить UML-диаграмму вариантов использования.
    \item Выделить основной вариант использования информационной системы (основной
        бизнес-процесс в предметной области).
    \item Разработать систему авторизации пользователей ИС.
    \item Разработать системную архитектуру ИС.
    \item Для всех вариантов использования разработать главные успешные сценарии и
        расширения к ним.
    \item Разработать системные UML-диаграммы последовательности для всех сценариев
        использованием MVC-паттерна.
    \item Разработать требования ко всем шаблонам для каждого варианта использования.
    \item Разработать инфологическую модель предметной области в форме UML-диаграммы классов.
    \item Разработать логическую модель будущей базы данных.
\end{enumerate}

\section*{Техническое задание на этап реализации}
Техническое задание на этап реализации состоит из следующих задач:
\begin{enumerate}
    \item 
    \item Реализовать разработанную на этапе проектирования информационную систему
        на языке Python в среде фреймворка Flask.
    \item Каждый вариант использования оформить, как blueprint.
    \item Доступ конечных и внешних пользователей к вариантам использования
        реализовать с помощью декораторов.
\end{enumerate}

\clearpage
\section*{Конечные пользователи информационной системы}
В данной системе пользователи разделяются на следующие категории:
\begin{enumerate}
    \item Кассир \textit{(внешний пользователь)}
        \begin{itemize}
            \item Авторизация
            \item Продажа билетов
        \end{itemize}
    \item Менеджер \textit{(внутренний пользователь)}
        \begin{itemize}
            \item Авторизация
            \item Продажа билетов
            \item Работа с запросами
        \end{itemize}
    \item Администратор \textit{(внутренний пользователь)}
        \begin{itemize}
            \item Авторизация
            \item Продажа билетов
            \item Работа с запросами
            \item Работа с отчетами
        \end{itemize}
\end{enumerate}

\begin{figure}[h!]
    \includegraphics[width=\linewidth]{build/img/scenarios.png}
    \caption{UML-диаграмма вариантов использования}
    \label{fig:scenarios}
\end{figure}

\clearpage

\section*{Главное меню}
Пункты главного меню содержат все варианты использования и пункт для выхода
из системы. При запуске системы управление передается контроллеру главного
меню.
Пункты главного меню для внутренних пользователей:
\begin{itemize}
    \item Покупка билетов;
    \item Работа с запросами;
    \item Работа с отчетами;
    \item Выход из аккаунта (системы).
    \item Пункты главного для внешних пользователей:
    \item Выход из аккаунта (системы).
\end{itemize}
\paragraph{Предусловия:} пользователь авторизировался.
\paragraph{Гарантия:} при наличии прав на использование выбранного пункта меню
пользователь переходит на страницу соответствующего варианта использования.
\paragraph{Минимальная гарантия:} пользователь остаётся в системе с сохранением состояния
сессии.
\paragraph{Сценарий работы главного меню:}
\begin{enumerate}
    \item Пользователь запускает сценарий;
    \item Система выдает главное меню;
    \item Пользователь выбирает один из пунктов (вариантов использования);
    \item Система перелает управление контроллеру соответствующего варианта использования.
\end{enumerate}
\clearpage
\paragraph{Исключения:}
\begin{enumerate}[label=\arabic*A.]
    \item Пользователь не авторизован.
\end{enumerate}

\begin{figure}[h!]
    \includegraphics[width=\linewidth]{build/img/main_menu.png}
    \caption{UML-диаграмма главного меню}
    \label{fig:scenarios}
\end{figure}

\paragraph{Требования к шаблонам HTML:} Динамический шаблон «Главное меню».
Шаблон показывает главное меню со всеми пунктами – вариантами использования и
информацию об авторизации.
Шаблон содержит ссылки:
\begin{itemize}
    \item На контроллер работы с запросами (адрес «blueprints/queries_menu»)
    \item На контроллер работы с отчётами (адрес «blueprints/ticket_report»)
    \item На контроллер бизнес-процесса (адрес «blueprints/ticket_cart»)
    \item На выход из системы (адрес «/logout»)
\end{itemize}
\clearpage

\section*{}

\end{document}
